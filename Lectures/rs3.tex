\documentclass[11pt]{amsart}


\newcommand{\R}{\mathbb R}
\newcommand{\Z}{\mathbb Z}
\newcommand{\la}{\langle}
\newcommand{\ra}{\rangle}

\newtheorem{theorem}{Theorem}
\newtheorem{definition}[theorem]{Definition}
\newtheorem{lemma}[theorem]{Lemma}
\newtheorem{remark}[theorem]{Remark}
\newtheorem{example}[theorem]{Example}

\begin{document}
\title{Simple and positive roots}
\author{Your name here}
\date{\today}
\maketitle

%\thispagestyle{empty}

{\Large  18.099 - 18.06 CI.} 

{Due on Monday, May 10 in class.} 

\vspace{1cm} 

{\it Write a paper proving the statements and working through the 
examples formulated below. Add your own 
examples, asides and discussions whenever needed. }



Let $V$ be a Euclidean space, that is  
a finite dimensional real linear space with a symmetric 
positive definite inner product $\la, \ra$. 

Recall that a root system in  
$V$ is a finite set $\Delta$ of nonzero 
elements of $V$ such that 
\begin{enumerate} 
\item{$\Delta$ spans $V$;}
\item{for all $\alpha \in \Delta$, the reflections  
$$ s_\alpha (\beta) = 
\beta -\frac{2 \la \beta, \alpha \ra }{\la \alpha, \alpha \ra}\alpha $$ 
map the set $\Delta$ to itself;}
\item{the number $\frac{2 \la \beta, \alpha \ra }{\la \alpha, \alpha \ra}$ 
is an integer for any $\alpha, \beta \in \Delta$.}
\end{enumerate} 
A root is an element of $\Delta$. 

Here are two examples of root systems in $\R^2$: 
\begin{example} \label{A1}
The root system of the type $A_1 \oplus A_1$ consists of 
the four vectors $\{ \pm e_1, \pm e_2 \}$ where $\{e_1,e_2 \}$ is an 
orthonormal basis in $\R^2$.
\end{example} 
\begin{example} \label{A2}
The root system of the type $A_2$ consists of the six vectors 
$\{e_i -e_j \}_{i\neq j}$ in the plane orthogonal to the line $e_1 +e_2 +e_3$ 
where $\{e_1, e_2, e_3\}$ is an orthonormal basis in $\R^3$.
Rewrite the vectors of this root system in a standard orthonormal basis of 
the plane and sketch it. 
\end{example}   


Since for any $\alpha \in \Delta$, $-\alpha$ is also in  $\Delta$, 
(see \cite{1}, Thm.8(1)),  
the number of elements in $\Delta$ is always 
greater than the dimension of $V$. The  
example of type $A_2$ above shows that even a subset of mutually 
noncollinear vectors in $\Delta$ might be too big to be linearly independent.  
In the present paper we would like  
to define a subset of $\Delta$ small enough to be a basis for $V$, yet 
large enough to contain the essential 
information about the geometric properties of $\Delta$. Here is a formal 
definition. 

\begin{definition} A subset $\Pi$ of $\Delta$ is a \emph{set of simple roots} 
(a \emph{simple root system}) in $\Delta$ if \begin{enumerate}
\item{ $\Pi$ is a basis for $V$;}
\item{Each root $\beta \in \Delta$ can be written as a linear 
combination of the elements of $\Pi$ with integer coefficients of the 
same sign, that is,
$$ \beta = \sum_{\alpha \in \Pi} m_\alpha \alpha $$
with all $m_\alpha \geq 0$ or all $m_\alpha \leq 0$.}
\end{enumerate}
 The root $\beta$ is 
\emph{positive} if the coefficients are nonnegative, 
and \emph{negative} otherwise. 
The set of all positive roots (the \emph{positive root system}) associated to 
$\Pi$ will be denoted $\Delta^+$. 
\end{definition} 

Below we construct a set $\Pi_t$ associated to an element $t \in V$ and 
a root system $\Delta$, and 
show that it satisfies the definition of a simple root system in $\Delta$. 

Let $\Delta $ be a root system in $V$, and 
let $t \in V$ be a vector such that $\langle t, \alpha \rangle \neq 0$ for 
all $\alpha \in \Delta$ (Check that such an element always exists). 
Set 
$$\Delta_t^+ = \{ \alpha \in \Delta : \langle t, \alpha \rangle >0 \}.$$ 
Let $\Delta_t^- = \{ -\alpha, \alpha \in \Delta_t^+ \}$. 
Check that $\Delta = \Delta_t^+ \cup \Delta_t^-$. 

\begin{definition} An element $\alpha \in \Delta_t^+$ is \emph{decomposable} 
if there exist $\beta, \gamma \in \Delta_t^+$ such that 
$\alpha = \beta +\gamma$. Otherwise $\alpha \in \Delta_t^+$ is 
\emph{indecomposable}.
\end{definition} 

Let $\Pi_t \subset \Delta_t^+$ be the set of all indecomposable elements 
in $\Delta_t^+$. 

The next three Lemmas prove the properties of $\Delta_t^+$  and $\Pi_t$.   

\begin{lemma} \label{pos} 
Any element in $\Delta_t^+$ can be written as a linear combination 
of elements in $\Pi_t$ with nonnegative integer coefficients. 
\end{lemma}
Hint: By contradiction. Suppose $\gamma$ is an element of $\Delta_t^+$ 
for which the Lemma is false and $\langle t, \gamma \rangle>0$ is minimal, 
and use that $\gamma$ is decomposable to get a contradiction. 

\begin{lemma} \label{angle} 
If $\alpha, \beta \in \Pi_t$, then $\langle \alpha, \beta \rangle \leq 0$. 
\end{lemma}
Hint: Use Thm. 10(1) in \cite{1} : if $\langle \alpha, \beta \rangle >0$, 
then $\alpha - \beta$ is a root or $0$. 

Add discussion: what does this result mean for the relative position 
of two simple roots? 

\begin{lemma} \label{basis}
Let $A$ be a subset of $V$ such that 
\begin{enumerate} 
\item{$\langle t, \alpha \rangle >0$ for all $\alpha \in A$;}
\item{$\langle \alpha, \beta \rangle \leq 0$ for all $\alpha, \beta \in A$.}
\end{enumerate}
Then the elements of $A$ are linearly independent. 
\end{lemma}
Hint: Assume the elements of $A$ are linearly dependent and split the 
nontrivial linear combination into two sums, 
with positive and negative coefficients. 
Let $\lambda = \sum m_\beta \beta = \sum n_\gamma \gamma$ with 
$\beta, \gamma \in A$ and all $ m_\beta , n_\gamma >0$. Show that 
$\langle \lambda, \lambda \rangle =0$. 

Now we are ready to prove the existence of a simple root set   
in any abstract root system.    

\begin{theorem} \label{simple}
For any $t \in V$ such that $\langle t, \alpha \rangle \neq 0$ for all 
$\alpha \in \Delta$, the set $\Pi_t$ constructed above is a set of simple 
roots, and $\Delta_t^+$ the associated set of positive roots. 
\end{theorem} 
Hint: Use lemmas \ref{pos}, \ref{angle}, \ref{basis}. 

The converse statement is also true (and much easier to prove): 
\begin{theorem} Let $\Pi$ be a set of simple roots in $\Delta$, and 
suppose that $t \in V$ is such that $\langle t, \alpha \rangle >0$ for all 
$\alpha \in \Pi$. Then $\Pi = \Pi_t$, and the associated set of positive 
roots $\Delta^+ = \Delta_t^+$. 
\end{theorem} 


\begin{example} Let $V$ be the 
$n$-dimensional subspace 
of $\R^{n+1}$ ($n \geq 1$) orthogonal to the line 
$e_1 + e_2 + \ldots + e_{n+1}$,
where $\{e_i\}_{i=1}^{n+1}$ is an orthonormal basis in $\R^{n+1}$. 
The root system $\Delta$ of the type $A_n$ in $V$ 
consists of all 
vectors $\{ e_i - e_j\}_{ i \neq j }$.  
 Check that $\Pi = \{ e_1 - e_2, e_2 - e_3, \ldots e_n -e_{n+1} \}$ is 
a set of simple roots, and $\Delta^+ = \{ e_i - e_j\}_{i<j}$ - the 
associated set of positive roots in $\Delta$. 
\end{example}  

\begin{example} \label{Cn}
The root system $\Delta$ 
of the type $C_n$ in $V=\R^n$ ($n \geq 2$) consists 
of all vectors $\{ \pm e_i \pm e_j\}_{i \neq j} \cup \{ \pm 2e_i \}$, 
where $\{e_i\}_{i=1}^n$ is an orthonormal basis in $\R^n$. Check that 
  $\Pi = \{ e_1 - e_2, e_2 - e_3, \ldots e_{n-1} -e_n, 2e_n \}$ is 
a set of simple roots, and $\Delta^+ = \{ e_i \pm e_j\}_{i<j}\cup \{2e_i\}$ 
- the associated set of positive roots in $\Delta$.   
\end{example} 

  
\begin{example} Let $V= \R^2$ and recall from \cite{1}, that for 
any two roots $\alpha, \beta$, 
$$ n(\alpha, \beta) \cdot n(\beta, \alpha) = 4 \cos^2(\phi),$$ 
where $n(\alpha, \beta) = 
\frac{2\la \beta, \alpha\ra}{\la \alpha, \alpha \ra }$, and 
$\phi$ is the angle between $\alpha$ and $\beta$.  Using Lemma 
\ref{angle}, find all possible angles between the simple roots in $\R^2$, 
and their relative lengths. Sketch the obtained pairs of vectors. 
Identify those that correspond to the root systems $A_1 \oplus A_1$, 
$A_2$ and $C_2$ discussed in Examples \ref{A1}, \ref{A2} and \ref{Cn} 
for $n=2$. In these three 
cases, describe and sketch the set of all elements 
$t \in V$ such that $\Pi_t = \Pi$ for a given $\Pi$. This set is the 
dominant Weyl chamber for $(\Delta, \Pi)$.  
\end{example}


\begin{thebibliography}{2}

\bibitem[1]{1} Your classmate, {\it Abstract root systems}, 
preprint, MIT, 2004. 


\end{thebibliography}


\end{document}

























