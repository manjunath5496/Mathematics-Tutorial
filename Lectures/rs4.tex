\documentclass[11pt]{amsart}


\newcommand{\R}{\mathbb R}
\newcommand{\Z}{\mathbb Z}
\newcommand{\la}{\langle}
\newcommand{\ra}{\rangle}

\newtheorem{theorem}{Theorem}
\newtheorem{definition}[theorem]{Definition}
\newtheorem{corollary}[theorem]{Corollary}
\newtheorem{remark}[theorem]{Remark}
\newtheorem{example}[theorem]{Example}

\begin{document}
\title{Properties of simple roots}
\author{Your name here}
\date{\today}
\maketitle

%\thispagestyle{empty}

{\Large  18.099 - 18.06 CI.} 

{Due on Monday, May 10 in class.} 

\vspace{1cm} 

{\it Write a paper proving the statements and working through the examples 
formulated below. Add your own 
examples, asides and discussions whenever needed. }



Let $V$ be a Euclidean space, that is 
a finite dimensional real linear space with a symmetric 
positive definite inner product $\la, \ra$. 

Recall that for a root system $\Delta$ in  
$V$,
a subset $\Pi \subset \Delta$ is a set of simple roots (a simple root system)
if \begin{enumerate}
\item{ $\Pi$ is a basis in $V$;}
\item{Each root $\beta \in \Delta$ can be written as a linear 
combination of elements of $\Pi$ with integer coefficients of the same sign,
i.e.
$$ \beta = \sum_{\alpha \in \Pi} m_\alpha \alpha $$
with all $m_\alpha \geq 0$ or all $m_\alpha \leq 0$.} 
\end{enumerate} 
 The root $\beta$ is 
positive if the coefficients are nonnegative, and negative otherwise. 
The set of all positive roots (positive root system) 
associated to $\Pi$ is denoted $\Delta^+$. 

Below we will assume that the root system $\Delta$ is reduced, that is, 
for any $\alpha \in \Delta, 2 \alpha \notin \Delta$.

\begin{theorem} \label{sim-pos} 
In a given $\Delta$, a set of simple roots $\Pi \subset \Delta$ 
and the associated set of positive roots 
$\Delta^+ \subset \Delta$ determine each other uniquely.
\end{theorem} 
Hint: easy. Use the explicit construction of $\Pi \subset \Delta^+$ given 
in \cite{3}.

The question of existence of sets of simple roots  for any abstract root 
system $\Delta$ is settled in \cite{3}.  Theorem \ref{sim-pos} 
shows that once $\Pi$ is chosen $\Delta^+$ is unique. 
In this paper we want to address the question of the possible choices for 
$\Pi \subset \Delta$. We start with a couple of examples. 

\begin{example} \label{A2}
 The root system of the type $A_2$ consists of the six vectors 
$\{e_i -e_j \}_{i\neq j}$ in the plane orthogonal to the line $e_1 +e_2 +e_3$ 
where $\{e_1, e_2, e_3\}$ is an orthonormal basis in $\R^3$.
Present the vectors of this root system in a standard orthonormal basis of 
the plane. Find possible simple root systems $\Pi \subset \Delta$ and 
the associated sets of positive roots $\Delta^+$, 
$\Pi \subset \Delta^+ \subset \Delta$.  
Check that any two simple root systems 
$\Pi \subset \Delta$ can be mapped to each other  by an orthogonal 
transformation (see \cite{1} for definition) of $V$, and that the same 
transformation maps the associated sets of positive roots.    
\end{example}   

\begin{example} \label{B2} 
Consider the root system of the type $B_2$ in $V= \R^2$: it consists of eight 
vectors $\{\pm e_1 \pm e_2, \pm e_1, \pm e_2 \}$. Find possible simple 
root systems $\Pi \subset \Delta$ and check that they can be obtained from 
any chosen one by an orthogonal transformation of $\R^2$. Check that 
the same transformation maps the associated sets of positive roots to each 
other. 
\end{example} 

We start working towards a result generalizing our observations. 
Recall the definition of a reflection associated to an element $\alpha \in V$ 
(cf. \cite{1}):
$$ s_\alpha(x) = x - \frac{2\langle x, \alpha \rangle}{\langle \alpha, \alpha 
\rangle}\alpha .$$ 
It is an orthogonal transformation of $V$. 

\begin{theorem} Let $\Pi \subset \Delta$ be a set of simple roots, associated 
to the set of positive roots $\Delta^+$. For any $\alpha \in \Delta$, 
the set obtained by reflection $s_\alpha(\Pi)$ is a simple root system   
with the associated positive root system $s_\alpha (\Delta^+)$. 
\end{theorem} 

To understand better the passage from $\Delta^+$ to $s_\alpha(\Delta^+)$, 
we consider the special case when $\alpha$ is a simple root. Then 
$\Delta^+$ and $s_\alpha(\Delta^+)$ differ by only one root: 

\begin{theorem} \label{sa}
Let $\Pi \subset \Delta$ be a simple root system, 
contained in a positive root set $\Delta^+$. If $\alpha \in \Pi$, then 
the reflection $s_\alpha$ maps the set $\Delta^+ \setminus \{\alpha\}$ to 
itself.     
\end{theorem} 

\begin{corollary} Any two positive root systems in $\Delta$ can be obtained 
from each other by a composition of reflections with respect to the roots 
in $\Delta$. 
\end{corollary} 

Hint: Let $\Delta^+_1 $ and $\Delta^+_2$ be two positive root systems. 
Recall that the negative roots $\Delta^-_i$ are the negatives of 
the elements in $\Delta^+_i$, $i=1,2$ (see \cite{3}). 
Use induction on the number of elements in the intersection 
$\Delta^+_1 \cap \Delta^-_2$. Theorem \ref{sa} provides a way to decrease 
this number by one. 

The statements above show that although a set of simple roots 
is not unique for a given $\Delta$, they are related to each other by 
a simple orthogonal transformation of the space $V$. In particular, 
the angles and relative lengths of simple roots in any two simple 
root systems in $\Delta$ are the same. 
The next theorem proves another useful 
property of simple roots. 

\begin{theorem} \label{seq}
Let $\Pi \subset \Delta^+$ be a simple and a positive 
root systems in $\Delta$.  
Any positive root $\beta \in \Delta^+$ can be written as a sum 
$$ \beta = \alpha_1 + \alpha_2 + \ldots + \alpha_k, $$
where $\alpha_i \in \Pi$ for all $i =1, \ldots , k$ (repetitions are allowed). 
Moreover, 
it can be done so that each partial sum 
$$ \alpha_1 + \ldots + \alpha_m, \;\; 1 \leq m \leq k $$ 
is also a root.      
\end{theorem} 

Hint: Choose $t \in V$ such that $\langle t, \alpha \rangle =1$ for 
all $\alpha \in \Pi$. Prove that such $t$ exists and that the number 
$r=\langle t, \beta \rangle $ 
is a positive integer for any $\beta \in \Delta^+$. Using Lemma 7 in 
\cite{3}, show that $\langle \alpha, \beta \rangle >0$ for some 
$\alpha \in \Pi$. 
Then proceed by induction on $r$. Theorem 10(1) in \cite{2} allows 
us to reduce $r$ by one.  

\begin{example} Let $\Delta$ be the root system in $V= \R^2$  
such that the angle between the simple roots is $\frac{5\pi}{6}$.
This condition determines $\Delta$ completely (this is the root system 
of the type $G_2$). 
 Construct and sketch the simple roots, 
positive roots, and the whole root system $\Delta$. Apply 
Theorem \ref{seq} in this case to present each positive root 
as a sum of simple roots. 
\end{example} 

Recall that two root systems $\Delta$ and $\Delta'$ are \emph{isomorphic} 
if there exists an linear automorphism of $V$ that maps 
$\Delta$ onto $\Delta'$ and preserves the integers
$\frac{2\la \beta, \alpha\ra}{\la \alpha, \alpha\ra}i$. 
A root system is irreducible if it 
cannot be decomposed as a disjoint union of two root systems 
$\Delta = \Delta' \cup \Delta''$ of smaller dimension, so that each 
element of $\Delta'$ is orthogonal to each element of $\Delta''$. 

\begin{example} Up to isomorphism, there are just three reduced irreducible 
root systems in $V=\R^3$, of the types $A_3$, $B_3$ and $C_3$ 
(see Example 5 in \cite{2}, Examples 10 and 11 in \cite{3} for 
definitions). Find the other possible reduced root systems in $V= \R^3$ (they 
can be represented as a union of two or more root systems in 
smaller dimensions). 
\end{example}
Hint: Note that the only reduced root system in $\R$ 
is of the type $A_1$. A classification of root systems in $\R^2$
can be carried out as indicated at the end of \cite{2}.


\begin{thebibliography}{2}

\bibitem[1]{1} Your classmate, {\it Reflections in a Euclidean space}, 
preprint, MIT, 2004.

\bibitem[2]{2} Your classmate, {\it Abstract root systems}, 
preprint, MIT, 2004. 

\bibitem[3]{3} Your classmate, {\it Simple and positive roots},
preprint, MIT, 2004.

\end{thebibliography}


\end{document}

























